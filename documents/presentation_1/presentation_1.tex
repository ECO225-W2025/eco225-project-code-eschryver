\documentclass[aspectratio=169]{beamer}
\setbeamertemplate{navigation symbols}{}
\usepackage{graphicx}
\usepackage{booktabs}
\usepackage{xcolor}
\newcommand {\framedgraphic}[2] {
    \begin{frame}{#1}
        \begin{center}
            \includegraphics[width=\textwidth,height=0.8\textheight,keepaspectratio]{#2}
        \end{center}
    \end{frame}
}
\usetheme{Madrid}
%%% Colors
\definecolor{ForestGreen}{RGB}{34,139,34}
\usecolortheme[named=ForestGreen]{structure}

\usepackage{amsthm}
\title[Minority Move-Ins and Home Appreciation]{"Blockbusting" in the 21st Century?: Minority Move-ins and Neighborhood Home Value Appreciation}
\author{Emerson Schryver}
\begin{document}
\begin{frame}
    \maketitle
\end{frame}
%%%%%%%%%%%%%%%%%%%%%%%%%%%%
\begin{frame}
\frametitle{Introduction}
\begin{itemize}
    \item Investigates whether minority move-ins suppress home-value appreciation.
    \item Uses loan data (Fannie Mae \& Freddie Mac) for move-ins, ACS (Census) data for normalization, and zip-code level Zillow ZHVI for home prices.
    \item Methodology:
    \begin{itemize}
        \item Select majority-white zip codes with minority move-ins (2009-2010).
        \item Track change in minority move-in share (2012-2013) as treatment.
        \item Analyze home price appreciation over following six years (until 2019).
    \end{itemize}
    \item Findings: The relationship between minority move-in share and home-value appreciation is unclear—while the treatment group shows lower appreciation, variation is extremely high.
\end{itemize}
\end{frame}
%%%%%%%%%%%%%%%%%%%%%%%%%%%%
\begin{frame}
    \frametitle{Background}
    \begin{itemize}
        \item Housing discrimination: long-running problem in US
        \item Common historical tactics: racial deed covenants, redlining, white flight, and blockbusting (Rothstein, 2017).
        \item Long-term effects:
        
        \small
        - Deed covenants improve relative neighborhood quality (Sood, Ehrman-Solberg, 2024), \\
        - Redlining localizes poverty (Appel, Nickerson, 2016), \\
        - Majority-Black neighborhoods maintain lower quality of opportunity (Chetty et al., 2014),
    
        \normalsize
        \item Modern discrimination persists:
        
        \small
        - Lending disparities (Quillian, Lee, Honoré, 2020), \\
        - Racial steering in real estate (Glenn, 2018)
        
    \end{itemize}
\end{frame}
%%%%%%%%%%%%%%%%%%%%%%%%%%%%
% \framedgraphic{Data}{project_files/project_40_0.png}
%%%%%%%%%%%%%%%%%%%%%%%%%%%%
\begin{frame}
    \frametitle{Data: Home Value Appreciation (\%) (difference to MSA average) (by year)}
        \centering
        \begin{tabular}{lrrrrrrrr}
            \toprule
            Year & count & mean & std & min & 25\% & 50\% & 75\% & max \\
            \midrule
            \textbf{2011} & 11376 & 0.0016 & 0.0375 & -0.2958 & -0.0149 & 0.0038 & 0.0213 & 0.2751 \\
            \textbf{2012} & 11376 & 0.0017 & 0.0587 & -0.4130 & -0.0258 & 0.0038 & 0.0330 & 0.3739 \\
            \textbf{2013} & 11376 & -0.0005 & 0.0782 & -0.4971 & -0.0384 & 0.0004 & 0.0395 & 0.6301 \\
            \textbf{2014} & 11376 & -0.0005 & 0.0962 & -0.6182 & -0.0480 & -0.0024 & 0.0442 & 0.8238 \\
            \textbf{2015} & 11376 & 0.0000 & 0.1192 & -0.6584 & -0.0599 & -0.0053 & 0.0506 & 1.1286 \\
            \textbf{2016} & 11376 & -0.0011 & 0.1439 & -0.7405 & -0.0726 & -0.0092 & 0.0550 & 1.4750 \\
            \textbf{2017} & 11376 & -0.0010 & 0.1724 & -0.8327 & -0.0873 & -0.0125 & 0.0610 & 1.7768 \\
            \textbf{2018} & 11376 & -0.0006 & 0.2089 & -0.9294 & -0.1023 & -0.0175 & 0.0696 & 2.3199 \\
            \textbf{2019} & 11376 & 0.0004 & 0.2312 & -1.0062 & -0.1133 & -0.0198 & 0.0783 & 2.7326 \\
            \bottomrule
        \end{tabular}
\end{frame}
%%%%%%%%%%%%%%%%%%%%%%%%%%%%
\begin{frame}
    \frametitle{Data: Minority Move-ins}
    \centering\begin{tabular}{lrrrr}
        \toprule
         & MMI (I) & MMI (T) & MI (I) & MI (T) \\
        \midrule
        \textbf{Count} & 33503 & 33503 & 33503 & 33503 \\
        \textbf{Mean} & 1 & 6 & 12 & 62 \\
        \textbf{Standard Deviation} & 4 & 20 & 37 & 148 \\
        \textbf{Minimum} & 0 & 0 & 0 & 0 \\
        \textbf{25th Percentile} & 0 & 0 & 0 & 0 \\
        \textbf{Median} & 0 & 0 & 0 & 0 \\
        \textbf{75th Percentile} & 0 & 2 & 2 & 46 \\
        \textbf{Maximum} & 113 & 620 & 641 & 1961 \\
        \textbf{Range} & 113 & 620 & 641 & 1961 \\
        \bottomrule
        \end{tabular}

        MMI = Minority Move-Ins, MI = Total Move-Ins
        
        I = Initial Period, T = Treatment Period
\end{frame}
%%%%%%%%%%%%%%%%%%%%%%%%%%%%
\framedgraphic{Results}{project_files/project_43_1.png}
%%%%%%%%%%%%%%%%%%%%%%%%%%%%
\framedgraphic{Results}{project_files/project_39_0.png}
%%%%%%%%%%%%%%%%%%%%%%%%%%%%
\framedgraphic{Results}{project_files/project_49_1.png}
%%%%%%%%%%%%%%%%%%%%%%%%%%%%
\begin{frame}
\frametitle{Conclusion}
\begin{itemize}
    \item There is little to no clear association between an increase in minority move-ins and changes in home values at the zip-code level.
    \item The analysis normalizes home value changes to the MSA and initial home values for a more accurate comparison.
    \item While controlling for initial home values and MSA-level appreciation, the data does not provide strong evidence that minority move-ins impact future home values.
    \item Due to data limitations, the extent to which minority move-ins influence home values remains inconclusive.
\end{itemize}
\end{frame}
\end{document}