\documentclass{article}
\usepackage{hyperref}
\usepackage[margin=1in]{geometry}

\title{"Blockbusting" in the 21st Century?: Minority Move-ins and Neighborhood Home Value Appreciation}
\author{Emerson Schryver\thanks{Undergraduate Student, University of Toronto, \href{mailto:emerson.schryver@mail.utoronto.ca}{emerson.schryver@mail.utoronto.ca}}}
\begin{document}
\maketitle
\begin{abstract}
    This paper investigates the link between minority move-ins and neighborhood home value appreciation. It analyzes

    % \textbf{JEL Codes:} J15, R23, R33
\end{abstract}
\newpage
\section{Introduction}
\indent Housing discrimination has, in one way or another, existed in the United States since independence. After reconstruction, several tactics became commonplace, including racial deed covenants, and redlining. Across the nation, there was evidence of ``white flight'', or the movement of whites out of neighborhoods with minorities due to fears over home value depreciation or other factors. Real-estate agents commonly attempted to abuse these fears with a tactic known as "blockbusting" in which they would spread fear over minority-move-ins leading to a fire sale of homes in a neighborhood (Rothstein, 2017).

Much has been researched about the modern-day effects of these past tactics, whether in deed covenants leading to improved relative neighborhood quality (Sood, Ehrman-Solberg, 2024), or redlining leading to localized areas of higher poverty (Appel, Nickerson, 2016), or simply lower quality of opportunity in majority-black neighborhoods (Chetty, et al, 2014). There is also evidence of modern-day tactics still occurring in the housing market, whether in lending markets (Quillian, Lee, Honoré, 2020), or in real estates continuing to practice "racial steering", the process of, whether knowing it or not, primarily showing people of minority groups neighborhoods that are also primarily of that minority group (Glenn, 2018). 

This paper seeks to contribute literature surrounding economic effects of racism  by analyzing the modern accuracy of the perceived link that 50's era blockbusting relied on -- do minority move-ins suppress home-value appreciation?

There is a long theoretical literature on discrimination, intiially starting with a model for taste-based discrimination by Gary Becker in 1957. There now many other models for discrimination, including most famously Ken Arrow's 1973 'statistical discriminination', but Becker's remains the most commonly used. There are also many theoretical frameworks for home sale matching, (Badarinza, Balasubramaniam, Ramadorai, 2024), and discriminiation in labor market matching (Combes et al, 2016), but there have been few attempts to discrimination theory in the housing market. The sole notable attempt was in (Combs, et al, 2015), which built a theoretical framework for racism in home sales and lease arrangements, and then empirically tested the lease framework. This has been extended to look at a case in Moscow, where much of the racial discirmination is overt (Avetian, 2022)

This paper contributes to this literature by applying this theoretical framework to home sales, and, in particular, analyzing the long-term effects on home values. I use loan data from Fannie Mae (FNMA) \& Freddie Mac (FHLMC) to provide data on
neighborhood move-ins, American Community Survey data to normalize, and
Zillow ZHVI data to show the change in home prices. My study has the
following methodology:

First, I select zip codes with move-ins during the 2009-2010 period
(Initial Period) that are majority white (we are uninterested in the
effect on majority-minority neighborhoods), and in metropolitan
statistical areas (necessary for appreciation normalization). I take the
``minority move-in share'', or the ratio of loans for new purchases made
in that zip code to minorities, and then look at those same zip codes
over the 2012-2013 period (Treatment Period) and analyze what
``treatment'' was applied (the difference in minority move-in share from
the previous period), which is our main explanatory variable. Next, we
analyze the association between the change in minority move-in-share and
the future appreciation in home prices, relative to their MSA average,
for the next 6 years (ending in 2019). This should inform us what the
effect of a sudden increase in minority move-ins is on home value
appreciation.

I find that the effect of an increase in minority move-in share on
home-value appreciation is unclear, relative to the metropolitan
statistical area. While the mean home value appreciation does decrease
in the treatment group relative to the control, there is an incredibly
high degree of variation in the data.
\section{Model}
\section{Method}
\section{Results}
\section{Conclusion}
\section{References}
\appendix
\section{Figures}
\section{Tables}
\end{document}