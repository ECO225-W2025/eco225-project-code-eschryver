\documentclass[aspectratio=169]{beamer}
\setbeamertemplate{navigation symbols}{}
\usepackage{graphicx}
\usepackage{xcolor}
\newcommand {\framedgraphic}[2] {
    \begin{frame}{#1}
        \begin{center}
            \includegraphics[width=\textwidth,height=0.8\textheight,keepaspectratio]{#2}
        \end{center}
    \end{frame}
}
\usetheme{Madrid}
%%% Colors
\definecolor{ForestGreen}{RGB}{34,139,34}
\usecolortheme[named=ForestGreen]{structure}

\usepackage{amsthm}
\title[Minority Move-Ins and Home Appreciation]{"Blockbusting" in the 21st Century?: Minority Move-ins and Neighborhood Home Value Appreciation}
\author{Emerson Schryver}
\begin{document}
\begin{frame}
    \maketitle
\end{frame}
\begin{frame}
\frametitle{Background}
\begin{itemize}
    \item Housing discrimination: long-running problem in US
    \item Common historical tactics: racial deed covenants, redlining, white flight, and blockbusting (Rothstein, 2017).
    \item Long-term effects:
    
    \small
    Deed covenants improve relative neighborhood quality (Sood, Ehrman-Solberg, 2024), 
    Redlining localizes poverty (Appel, Nickerson, 2016),
    Majority-Black neighborhoods still have lower quality of opportunity (Chetty et al., 2014),

    \normalsize
    \item Modern discrimination persists:
    
    \small
    Lending disparities (Quillian, Lee, Honoré, 2020).
    Racial steering in real estate (Glenn, 2018).
    
\end{itemize}
\end{frame}
%%%%%%%%%%%%%%%%%%%%%%%%%%%%
\begin{frame}
\frametitle{Introduction}
\begin{itemize}
    \item Investigates whether minority move-ins suppress home-value appreciation.
    \item Uses loan data (Fannie Mae \& Freddie Mac), ACS (Census) data for normalization, and Zillow ZHVI for home prices.
    \item Methodology:
    \begin{itemize}
        \item Select majority-white zip codes with minority move-ins (2009-2010).
        \item Track change in minority move-in share (2012-2013) as treatment.
        \item Analyze home price appreciation over following six years (until 2019).
    \end{itemize}
    \item Findings: The relationship between minority move-in share and home-value appreciation is unclear—while the treatment group shows lower appreciation, variation is extremely high.
\end{itemize}
\end{frame}
%%%%%%%%%%%%%%%%%%%%%%%%%%%%
\framedgraphic{Data}{project_files/project_40_0.png}
%%%%%%%%%%%%%%%%%%%%%%%%%%%%
\begin{frame}
\frametitle{Results}
\begin{itemize}
    \item Relatively inconclusive
    \item Some indication of suppression of home values but not significant
\end{itemize}
\end{frame}
%%%%%%%%%%%%%%%%%%%%%%%%%%%%
\framedgraphic{Results}{project_files/project_43_1.png}
%%%%%%%%%%%%%%%%%%%%%%%%%%%%
\framedgraphic{Results}{project_files/project_46_0.png}
%%%%%%%%%%%%%%%%%%%%%%%%%%%%
\framedgraphic{Results}{project_files/project_49_1.png}
%%%%%%%%%%%%%%%%%%%%%%%%%%%%
\begin{frame}
\frametitle{Conclusion}
\begin{itemize}
    \item There is little to no clear association between an increase in minority move-ins and changes in home values at the zip-code level.
    \item The analysis normalizes home value changes to the MSA and initial home values for a more accurate comparison.
    \item While controlling for initial home values and MSA-level appreciation, the data does not provide strong evidence that minority move-ins impact future home values.
    \item Due to data limitations, the extent to which minority move-ins influence home values remains inconclusive.
\end{itemize}
\end{frame}
\end{document}